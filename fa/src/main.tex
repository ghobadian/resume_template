%%%%%%%%%%%%%%%%%
% This is an sample CV template created using altacv.cls
% (v1.7.2, 28 August 2024) written by LianTze Lim (liantze@gmail.com). Compiles with pdfLaTeX, XeLaTeX and LuaLaTeX.
%
%% It may be distributed and/or modified under the
%% conditions of the LaTeX Project Public License, either version 1.3
%% of this license or (at your option) any later version.
%% The latest version of this license is in
%%    http://www.latex-project.org/lppl.txt
%% and version 1.3 or later is part of all distributions of LaTeX
%% version 2003/12/01 or later.
%%%%%%%%%%%%%%%%

%% Use the "normalphoto" option if you want a normal photo instead of cropped to a circle
% \documentclass[10pt,a4paper,normalphoto]{altacv}

\documentclass[10pt,a4paper,ragged2e,withhyper]{altacv}
%% AltaCV uses the fontawesome5 and simpleicons packages.
%% See http://texdoc.net/pkg/fontawesome5 and http://texdoc.net/pkg/simpleicons for full list of symbols.

% Change the page layout if you need to
\geometry{left=1.25cm,right=1.25cm,top=1.5cm,bottom=1.5cm,columnsep=1.2cm}

% The paracol package lets you typeset columns of text in parallel
\usepackage{paracol}


% Change the font if you want to, depending on whether
% you're using pdflatex or xelatex/lualatex
% WHEN COMPILING WITH XELATEX PLEASE USE
% xelatex -shell-escape -output-driver="xdvipdfmx -z 0" main.tex
\usepackage{bidi}
\setRTL
\usepackage{xepersian}
\settextfont[Path=./fonts/, Extension=.ttf]{b_nazanin}
\newfontfamily\titr[Path=./fonts/, Extension=.ttf]{b_titr}
\newfontfamily\nazanin[Path=./fonts/, Extension=.ttf]{b_nazanin}

% Change the colours if you want to
\definecolor{SlateGrey}{HTML}{2E2E2E}
\definecolor{LightGrey}{HTML}{666666}
\definecolor{DarkPastelRed}{HTML}{450808}
\definecolor{PastelRed}{HTML}{8F0D0D}
\definecolor{GoldenEarth}{HTML}{E7D192}
\colorlet{name}{black}
\colorlet{tagline}{PastelRed}
\colorlet{heading}{DarkPastelRed}
\colorlet{headingrule}{GoldenEarth}
\colorlet{subheading}{PastelRed}
\colorlet{accent}{PastelRed}
\colorlet{emphasis}{SlateGrey}
\colorlet{body}{LightGrey}



% Change some fonts, if necessary
\renewcommand{\namefont}{\Huge\rmfamily\bfseries}
\renewcommand{\personalinfofont}{\footnotesize}
%\renewcommand{\cvsectionfont}{\LARGE\rmfamily\bfseries}
\renewcommand{\cvsubsectionfont}{\large\bfseries}


% Change the bullets for itemize and rating marker
% for \cvskill if you want to
\renewcommand{\cvItemMarker}{{\small\textbullet}}
\renewcommand{\cvRatingMarker}{\faCircle}
% ...and the markers for the date/location for \cvevent
% \renewcommand{\cvDateMarker}{\faCalendar*[regular]}
% \renewcommand{\cvLocationMarker}{\faMapMarker*}


% If your CV/résumé is in a language other than English,
% then you probably want to change these so that when you
% copy-paste from the PDF or run pdftotext, the location
% and date marker icons for \cvevent will paste as correct
% translations. For example Spanish:
% \renewcommand{\locationname}{Ubicación}
% \renewcommand{\datename}{Fecha}


%% Use (and optionally edit if necessary) this .tex if you
%% want to use an author-year reference style like APA(6)
%% for your publication list
% \input{pubs-authoryear.tex}

%% Use (and optionally edit if necessary) this .tex if you
%% want an originally numerical reference style like IEEE
%% for your publication list
\usepackage[backend=biber,style=ieee,sorting=ydnt]{biblatex}
%% For removing numbering entirely when using a numeric style
\setlength{\bibhang}{1.25em}
\DeclareFieldFormat{labelnumberwidth}{\makebox[\bibhang][l]{\itemmarker}}
\setlength{\biblabelsep}{0pt}
\defbibheading{pubtype}{\cvsubsection{#1}}
\renewcommand{\bibsetup}{\vspace*{-\baselineskip}}


\begin{document}
\name{\titr{محمدصادق قبادیان}}
\tagline{\titr{توسعه دهنده جاوا}}
%% You can add multiple photos on the left or right
\photoR{2.8cm}{formal_pic_square}
% \photoL{2.5cm}{Yacht_High,Suitcase_High}

\personalinfo{%
  % Not all of these are required!
  \lr{\email{msghobadian80@gmail.com}}
  \lr{\nazanin{\phone{0933-256-4335}}}
  % \mailaddress{Åddrésş, Street, 00000 Cóuntry}
  %\lr{\nazanin{\location{تهران}}}
  \lr{\homepage{msghobadian.ir}}
  \lr{\linkedin{ghobadian}}
  \lr{\github{ghobadian}}
  
  
  % \twitter{@twitterhandle}
  % \xtwitter{@x-handle}
  
  
  % \orcid{0000-0000-0000-0000}
  %% You can add your own arbitrary detail with
  %% \printinfo{symbol}{detail}[optional hyperlink prefix]
  % \printinfo{\faPaw}{Hey ho!}[https://example.com/]

  %% Or you can declare your own field with
  %% \NewInfoFiled{fieldname}{symbol}[optional hyperlink prefix] and use it:
  % \NewInfoField{gitlab}{\faGitlab}[https://gitlab.com/]
  % \gitlab{your_id}
  %%
  %% For services and platforms like Mastodon where there isn't a
  %% straightforward relation between the user ID/nickname and the hyperlink,
  %% you can use \printinfo directly e.g.
  % \printinfo{\faMastodon}{@username@instace}[https://instance.url/@username]
  %% But if you absolutely want to create new dedicated info fields for
  %% such platforms, then use \NewInfoField* with a star:
  % \NewInfoField*{mastodon}{\faMastodon}
  %% then you can use \mastodon, with TWO arguments where the 2nd argument is
  %% the full hyperlink.
  % \mastodon{@username@instance}{https://instance.url/@username}
}

\makecvheader
%% Depending on your tastes, you may want to make fonts of itemize environments slightly smaller
% \AtBeginEnvironment{itemize}{\small}

%% Set the left/right column width ratio to 6:4.
\columnratio{0.4}

% Start a 2-column paracol. Both the left and right columns will automatically
% break across pages if things get too long.


\begin{paracol}{2}

\cvsection{\titr{مهارت ها}}

\rl{
    \cvtag{\lr{Java}}
    \cvtag{\lr{Spring}}
    \cvtag{\lr{Git}}
    \cvtag{\lr{SQL}}
    \cvtag{\lr{OOP}}
    \\
    \cvtag{\lr{CI/CD}}
    \cvtag{\lr{TDD}}
    \cvtag{\lr{Redis}}
    \cvtag{\lr{MongoDB}}
}

\divider\smallskip
\rl{
    \cvtag{حل مسئله}
    \cvtag{کار گروهی}
    \cvtag{پشتکار}
    \\
    \cvtag{نظم}
    \cvtag{مدیریت زمان}
    \cvtag{اعتماد به نفس}
}


\medskip %% Use \smallskip, \medskip, \bigskip, \vspace etc to make adjustments.

\cvsection{\titr{پروژه ها}}

\cvevent{\href{https://github.com/ghobadian/golestan}{سامانه مدیریت دروس}}{متن باز}{}{}
توسعه سیستم انتخاب واحد و نمره دهی برای دانشجویان و اساتید دانشگاه ها

\divider

\cvevent{\href{https://github.com/ghobadian/SimpleMessanger}{پیام رسان}}{متن باز}{}{}
طراحی یک برنامه چت آنلاین با استفاده از 
\\
\lr{socket programming}
 و معماری کلاینت-سرور

\medskip

\cvsection{\titr{مجوز ها}}
\cvevent{\href{https://www.coursera.org/account/accomplishments/verify/16C32QLBKURP}{\lr{Building Scalable Java Microservices with Spring Boot and Spring Cloud}}}{\lr{Google Cloud}}{مرداد ۱۴۰۳}{}
\divider

\cvevent{\href{https://university.redis.com/certificates/a6d6561bb5844056911f177966a74d3b}{\lr{Redis for Java Developers}}}{\lr{Redis University}}{شهریور ۱۴۰۳}{}

\divider

\cvevent{\href{https://maktabkhooneh.org/certificates/MK-9826SE/}{مبانی شبکه}}{مکتب خونه}{مرداد ۱۴۰۳}{}

\divider

\cvevent{\href{https://www.sololearn.com/de/certificates/CC-DPFVE4JV}{\lr{Java Intermediate}}}{\lr{Sololearn}}{مرداد ۱۴۰۳}{}


\divider

\cvevent{\href{https://www.hackerrank.com/certificates/21a5a3247803}{\lr{SQL Intermediate}}}{\lr{HackerRank}}{مرداد ۱۴۰۳}{}

% use ONLY \newpage if you want to force a page break for
% ONLY the current column
\newpage

%% Switch to the right column. This will now automatically move to the second
%% page if the content is too long.
\switchcolumn

\cvsection{\titr{درباره من}}
\nazanin
مسلط بر جاوا و 
\lr{OOP}
\\
تسلط به 
\lr{Spring Framework (Core, Boot, Data, AOP, Security, Cloud)}
\\
توانایی کار با 
\lr{Git}
\\
آشنا با 
\lr{SQL، JDBC, ORM (JPA, Hibernate)}
\\
آشنا به فرایند 
\lr{CI/CD}
به کمک 
\lr{docker}
 در 
\lr{Gitlab}
\\
 توانایی اتوماسیون به کمک 
\lr{Bash}
  یا 
\lr{Python}
\\
  پایبند به اصول 
\lr{SOLID}
 کد تمیز و 
\lr{Design Pattern}
  ها
\\
  آشنا با معماری های 
\lr{MVC}
   و 
\lr{Microservices}
\\
توانایی حل مسئله و تفکر الگوریتمی
\\
مشتاق برای کار گروهی

\cvsection{\titr{سوابق شغلی}}
\nazanin
\cvevent{توسعه دهنده جاوا}{\href{https://sobhan.tech}{بشیر سبحان}}{تیر ۱۴۰۱ - آبان ۱۴۰۲ (۱ سال و ۴ ماه)}{اصفهان، ایران}
- توسعه و افزودن فیچر های جدید به
\lr{zarebin.ir}\\
- 
پیاده سازی نسخه های اولیه 
\lr{shopiway.ir}\\
- 
بهبود عملکرد کد های قدیمی به کمک کاهش 
\lr{SQL}
 کوئری های تکراری

\cvsection{\titr{سوابق تحصیلی}}

\cvevent{کارشناسی مهندسی کامپیوتر}{\href{https://en.kntu.ac.ir}{دانشگاه صنعتی خواجه نصیرالدین طوسی}}{مهر ۱۴۰۰ - الان}{}
به دانشجویان سال اول کمک شد تا مبانی برنامه نویسی و برنامه نویسی شی گرا را بهتر درک کنند

\end{paracol}

\end{document}